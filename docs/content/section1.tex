\section{Описание}

Данная лабораторная работа была разделена на два варианта реализации утилит.

\subsection{Первый вариант}

Данный вариант предусматривает, что создаются два потока (\textit{pthread}): первый читает пакеты при помощи Raw Socket (OSI L2) с интерфейса, проверяет параметры пакета и для подходящих по заданным параметрам, передаёт статистику во второй поток. Второй суммирует статистику и отдаёт её по запросу извне.

Передача статистики пользователю осуществляется через \textit{POSIX Message Queues}.

Передача статистики между потоками осуществляется через каналы (\textit{pipe}).

\subsection{Второй вариант}

Эта утилита создаёт два потока (\textit{pthread}). Первый поток читает пакеты при помощи Raw Socket (OSI L2) с интерфейса, проверяет параметры пакета и для подходящих по заданным параметрам, суммирует статистику. Второй поток отдаёт её по запросу извне.

Передача статистики пользователю осуществляется через \textit{POSIX Message Queues}.

\newpage

Для работы захвата пакетов обязательно нужно указать сетевой интерфейс, с которого будут считываться пакеты.

В статистике отражается количество принятых пакетов и суммарное количество байт в этих пакетах. Также существует возможность фильтрации пакетов по следующим параметрам:
\begin{enumerate}
    \item IP-адрес источника
    \item IP-адрес назначения
    \item Порт источника
    \item Порт назначения
\end{enumerate}

\subsection{Структура исходников}

Весь исходный код данного проекта содержится в папке \verb|src/|:

\begin{lstlisting}
src/
    basic.c
    colors.h
    Makefile
    ps-scanner-1.c
    ps-scanner-2.c
    ps-scanner.h
    ps-stats.c
\end{lstlisting}

В паке \verb|src/| содержатся два варианта реализации утилиты для сбора статистики: \verb|ps-scanner-1.c| -- первый варинат реализации, \verb|ps-scanner-2.c| -- второй вариант реализации.

\subsubsection*{Описание файлов}

\verb|ps-scanner-1.c|:\\
Данный файл содержит системную утилиту, которая собирает информацию о пакетах (первым вариантом реализации) и передаёт её пользователю через \textit{POSIX Message Queues}.

\newpage

\verb|ps-scanner-2.c|:\\
Данный файл содержит системную утилиту, которая собирает информацию о пакетах (вторым вариантом реализации) и передаёт её пользователю через \textit{POSIX Message Queues}.

\linespace

\verb|ps-scanner.h|:\\
Заголовочный файл для \verb|ps-scanner-1.c| и \verb|ps-scanner-2.c|, который содержит прототипы функций.

\linespace

\verb|basic.c|\\
В данный файл вынесены общие функции, которые не связаны логикой с \verb|ps-scanner-1.c| и \verb|ps-scanner-2.c|.

\linespace

\verb|ps-stats.c|:\\
В данном файле содержатся функции, которые выводят статистику для пользователя. Данные приходят от \verb|ps-scanner| через POSIX Message Queues.

\linespace

\verb|colors.h|:\\
Данный файл содержит константы цветов для более приятного пользовательского интерфейса.
