\section{Описание}

Данная лабораторная работа разделена на две части: \textit{ps-scanner} и \textit{ps-stats}.

\subsection{Сбор статистики: ps-scanner}

Эта утилита создаёт два потока (\textit{pthread}). Первый поток читает пакеты при помощи Raw Socket (OSI L2) с интерфейса, проверяет параметры пакета и для подходящих по заданным параметрам, суммирует статистику. Второй поток отдаёт её по запросу извне.

Передача статистики пользователю осуществляется через POSIX Message Queues.

\subsection{Вывод статистики: ps-stats}

В статистике отражается количество принятых пакетов и суммарное количество байт в этих пакетах. Также существует возможность фильтрации пакетов по следующим параметрам:
\begin{enumerate}
    \item IP-адрес источника
    \item IP-адрес назначения
    \item Порт источника
    \item Порт назначения
\end{enumerate}

\subsection{Структура исходников}

Весь исходный код данного проекта содержится в папке \verb|src/|:

\begin{lstlisting}[language=bash]
src/
    basic.c
    colors.h
    main.c
    main.h
    Makefile
    recipient.c
\end{lstlisting}

\linespace

\indent \verb|main.c|:\\
Данный файл содержит системную утилиту, которая собирает информацию о пакетах и передаёт её пользователю через POSIX Message Queues.

\linespace

\indent \verb|main.h|:\\
Заголовочный файл для \verb|main.c|, который содержит прототипы функций \verb|main.c|.

\linespace

\indent \verb|basic.c|:\\
В данный файл вынесены общие функции, которые не связаны логикой с \verb|main.c|.

\linespace

\indent \verb|colors.h|:\\
Данный файл содержит константы цветов для более приятного пользовательского интерфейса.

\linespace

\indent \verb|recipient.c|:\\
В данном файле содержатся функции, которые выводят статистику для пользователя. Данные приходят от \verb|main| через POSIX Message Queues.
