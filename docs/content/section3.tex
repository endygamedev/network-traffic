\section{Запуск}
Установленный deb-пакет содержит три утилиты:
\begin{enumerate}
    \item \verb|ps-scanner-1| -- утилита для сбора статистики (первый вариант реализации);
    \item \verb|ps-scanner-2| -- утилита для сбора статистики (второй вариант реализации);
    \item \verb|ps-stats| -- пользовательская утилита, которая предоставляет собранную статистику.
\end{enumerate}

\subsection{ps-scanner-1 и ps-scanner-2}
\verb|ps-scanner-1| и \verb|ps-scanner-2| получают на вход 1 обязательный аргмент и 4 опциональных аргумента, которые помогают отфильтровать нужные пакеты:
\begin{itemize}
    \item \verb|-ni| или \verb|--interface| -- \footnotesize\textit{!(обязательный аргумент)} \normalsizeсетевой интерфейс, с которого будут считываться пакеты.
    \item \verb|-ips| или \verb|--ip_source| -- IP-адрес источника отслеживаемых пакетов;
    \item \verb|-ipd| или \verb|--ip_dest| -- IP-адрес назначения отслеживаемых пакетов;
    \item \verb|-ps| или \verb|--port_source| -- порт источника отслеживаемых пакетов;
    \item \verb|-pd| или \verb|--port_dest| -- порт назначения отслеживаемых пакетов.
\end{itemize}

\linespace

Запуск \verb|ps-scanner-1| и \verb|ps-scanner-2|:
\begin{lstlisting}
$ sudo ps-scanner-1 -ni <interface name> -ips <IP source> -ipd <IP dest> -ps <port source> -pd <port dest>
\end{lstlisting}

\begin{lstlisting}
$ sudo ps-scanner-2 -ni <interface name> -ips <IP source> -ipd <IP dest> -ps <port source> -pd <port dest>
\end{lstlisting}

\newpage

Пример запуска \verb|ps-scanner-1| (аналогично для \verb|ps-scanner-2|):
\begin{lstlisting}
$ sudo ps-scanner-1 -ni lo -ipd 192.168.1.2 -pd 9999
\end{lstlisting}
В данном примере отбираются только те пакеты с сетевого интерфейса \verb|lo|, у которых IP-адрес назначения -- 192.168.1.2 и порт назначения -- 9999.

\linespace

Также в файл \verb|/var/log/ps-scanner.log| записываются все пакеты, которые прошли этап отбора, чтобы можно было проверить результат работы программы.

\linespace

Список всех доступных сетевых интерфейсов можно узнать следующим способом:
\begin{lstlisting}
$ netstat -i
\end{lstlisting}
или
\begin{lstlisting}
$ ip link show
\end{lstlisting}

\subsection{ps-stats}

Данная утилита идёт без аргументов и связывается с утилитой \verb|ps-scanner-1| или \verb|ps-scanner-2| по названию (идентификатору) очереди (Message Queue Name).

\verb|ps-stats| запускается после \verb|ps-scanner-1| или \verb|ps-scanner-2|, если попытаться выводить статистику без предварительного сбора, то будет напечатано сообщение с ошибкой.

\linespace

Запуск \verb|ps-stats|:
\begin{lstlisting}
$ sudo ps-stats
\end{lstlisting}

\newpage

\subsection{Скрипты}

Для проверки работоспособности утилит можно воспользоваться файлами из папки \verb|tests/|. Они создают локальный UDP-сервер и отправляют сообщения (пакеты) пользователя.

\begin{lstlisting}
tests/
    Makefile
    nc-client
    nc-server
    ps-scanner-test
    ps-stats-test
\end{lstlisting}

\linespace

\verb|ps-scanner-test|:
\begin{lstlisting}
#!/bin/bash
sudo ps-scanner-$1 -ni lo -ipd $(hostname -I | awk '{print $1}') -pd 9999
\end{lstlisting}
Данный скрипт принимает 1 обязательный позиционный аргумент, который определяет какую реализацию сбора статистики необходимос запустить (1 или 2). Запускается сканер на сетевом интерфейса \verb|lo| (Loopback interface), который отбирает только те пакеты, у которых IP-адрес назначения \verb|localhost|, а порт назначения \verb|9999|.

\linespace

Также если не собирать \verb|deb|-пакет, то можно протестировать данную утилиту из исходников:
\begin{lstlisting}
#!/bin/bash
sudo ../src/ps-scanner-$1 -ni lo -ipd $(hostname -I | awk '{print $1}') -pd 9999
\end{lstlisting}

\linespace

\verb|ps-stats|:
\begin{lstlisting}
#!/bin/bash
sudo ps-stats
\end{lstlisting}
В данном скрипте запускается вывод собранной статистики \verb|ps-scanner|.

\newpage

Также если не собирать \verb|deb|-пакет, то можно протестировать данную утилиту из исходников:
\begin{lstlisting}
#!/bin/bash
sudo ../src/ps-stats
\end{lstlisting}

\linespace

\verb|nc-server|:
\begin{lstlisting}
#!/bin/bash
nc -u -l 9999
\end{lstlisting}
С помощью утилиты \verb|netcat| запускается UDP-сервер на порте 9999, который ожидает сообщений от клиентов.

\linespace

\verb|nc-client|:\\
Данная утилита принимает 4 опциональных аргумента:
\begin{itemize}
    \item \verb|-c| -- количество посланных пользователем сообщений;
    \item \verb|-m| -- текст сообщения, которое будет послано;
    \item \verb|-s| -- IP-адрес сервера;
    \item \verb|-p| -- порт сервера.
\end{itemize}
То есть после запуска данной программы на сервер \verb|$count|-раз отправится сообщение \verb|$message|.

\subsection{Тесты}

Итак, для тестирования, с помощью \verb|netcat|, создаётся локальный UDP-сервер, на который посылаются UDP-пакеты (сообщения) клиентов.

\begin{enumerate}
\item Для начала стоит перейти в папку с тестом:
\begin{lstlisting}
$ cd ./tests/
\end{lstlisting}

\item Затем в запускается UDP-сервер:
\begin{lstlisting}
$ ./nc-server
\end{lstlisting}

\newpage

\item После чего можно запустить \verb|ps-scanner-1|:
\begin{lstlisting}
$ ./ps-scanner-test 1   # first implementation test
\end{lstlisting}
или \verb|ps-scanner-2|:
\begin{lstlisting}
$ ./ps-scanner-test 2   # second implementation test
\end{lstlisting}

\item Далее стоит запустить \verb|ps-stats|:
\begin{lstlisting}
$ ./ps-stats
\end{lstlisting}

\item После чего можно отправлять различные пакеты на сервер и смотреть как отображается статистика:
\begin{lstlisting}
$ ./nc-client
\end{lstlisting}

\item Также собранные пакеты можно посмотреть:
\begin{lstlisting}
$ cat /var/log/ps-scanner.log
\end{lstlisting}

\end{enumerate}

\linespace

Видеопример последовательного запуска программ предстален в репозитории GitHub.
