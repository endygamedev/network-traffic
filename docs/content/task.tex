\section*{Постановка задачи}
\addcontentsline{toc}{section}{Постановка задачи}

\textit{Цель задачи:} создать набор программного обеспечения, который мог бы собирать и отображать статистику по трафику на заданном сетевом интерфейсе.

\linespace

\textit{Требования:}
\begin{enumerate}[wide, noitemsep]
    \item ПО должно работать на ПК под управлением Debian GNU\textbackslash Linux (версии 10 и новее).
    \item Для реализации использовать язык программирования C.
    \item Сборка должна осуществляться GNU Toolchain.
    \item Дистрибуция должна осуществляться при помощи deb-пакета.
    \item Сбор статистики должен вестись только по входящим UDP пакетам.
    \item Должна быть реализована возможность указывать конкретные параметры учитываемых в статистике пакетов:
    \begin{enumerate}[wide=\dimexpr\parindent+1.25cm, noitemsep]
        \item IP-адрес источника
        \item IP-адрес назначения
        \item Порт источника
        \item Порт назначения
    \end{enumerate}
    \item В статистике должно присутствовать количество принятых пакетов и суммарное количество байт в этих пакетах.
\end{enumerate}

\linespace

ПО организовать в виде двух отдельных утилит: первая читает данные с сетевого интерфейса и собирает статистику по пакетам, вторая при запуске получает собранную статистику у первой утилиты и выводит её на экран.

\linespace

Утилита для сбора статистики. Нужно реализовать 2 варианта данной утилиты:
\begin{enumerate}[wide, noitemsep]
    \item Два потока (pthread): первый читает пакеты при помощи Raw Socket (OSI L2) с интерфейса, проверяет параметры пакета и для подходящих по заданным параметрам, передаёт статистику во второй поток. Второй суммирует статистику и отдаёт её по запросу извне.
    \item Два потока (pthread): первый читает пакеты при помощи Raw Socket (OSI L2) с интерфейса, проверяет параметры пакета и для подходящих по заданным параметрам, суммирует статистику. Второй отдаёт её по запросу извне.
\end{enumerate}

\linespace

Самостоятельно провести профилирование обоих вариантов, оценить какой вариант эффективнее, с каким вариантом можно обеспечить большую пропускную способность.

В обоих вариантах: передача данных между потоками осуществляется любым способом, на усмотрение разработчика.

Утилита для вывода статистики на экран: запрашивает статистику у первой утилиты через ubus или через POSIX Message Queues. Рекомендация: попробовать реализовать оба варианта.

\linespace

Программное обеспечение должно сопровождаться документацией, содержащей следующие разделы:
\begin{enumerate}[wide, noitemsep]
    \item Описание - общая информация, что и как делает ПО.
    \item Сборка - инструкции по сборке ПО из исходников: что установить в систему,
какой командой запустить сборку, что должно получиться в итоге.
    \item Запуск - как запустить ПО, как подать трафик на интерфейс, чтобы убедиться в
корректности работы, что пользователь программ должен увидеть на экране.
    \item Результаты профилирования двух реализованных вариантов утилиты для сбора статистики.
    \item Авторство и лицензия - указать имя и электронную почту автора, указать лицензию.
\end{enumerate}

Результат работы: архив git репозитория, содержащего исходники ПО и сопроводительную документацию.

\linespace

Вспомогательная информация:
\begin{itemize}
    \item The Linux Programming Interface: \url{https://man7.org/tlpi/}
    \item GCC: \url{https://gcc.gnu.org}
    \item pthread: \url{https://man7.org/linux/man-pages/man7/pthreads.7.html}
    \item Message Queues: \url{https://man7.org/linux/man-pages/man7/mq_overview.7.html}
    \item Raw Sockets: \url{https://man7.org/linux/man-pages/man7/raw.7.html}
    \item ubus: \url{https://openwrt.org/docs/techref/ubus}
    \item Сборка deb-пакета: \url{https://www.debian.org/doc/devel-manuals#debmake-doc}
    \item Профилирование: \url{https://www.brendangregg.com/}
    \item Git: \url{https://git-scm.com/book/en/v2}
\end{itemize}
